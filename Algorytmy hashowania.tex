\documentclass{article}
\usepackage[utf8]{inputenc}
\usepackage[T1]{fontenc}
\usepackage{polski}
\usepackage{amsmath}
\usepackage{amsfonts}
\usepackage{amssymb}
\usepackage{graphicx}
\usepackage{hyperref}
\usepackage{listings}
\usepackage{color}

\title{Algorytmy Hashowania}
\author{Kamil Filipiński, 66140}
\date{}

\begin{document}

\maketitle

\section{Omówienie problemu}
Algorytmy hashowania są fundamentalnymi narzędziami w kryptografii i innych dziedzinach informatyki. Umożliwiają one efektywne przechowywanie, przetwarzanie i zabezpieczanie danych. Hashowanie jest procesem przekształcania danych wejściowych (np. hasła) na ciąg znaków o ustalonej długości, zwany hashem. Hashe są wykorzystywane m.in. do przechowywania haseł, sprawdzania integralności danych oraz w kryptowalutach. W tym kontekście kluczowe jest zapewnienie, aby hash był odporny na ataki kryptograficzne, takie jak ataki brute-force i kolizje.

\section{Przegląd algorytmów hashowania}

\subsection{B-Crypt}
\textbf{Zastosowania:} B-Crypt jest najczęściej używany do bezpiecznego przechowywania haseł. Jego odporność na ataki brute-force i rainbow table sprawia, że jest preferowany w aplikacjach wymagających wysokiego poziomu bezpieczeństwa.

\textbf{Działanie:}
\begin{itemize}
    \item B-Crypt bazuje na algorytmie Blowfish i jest specjalnie zaprojektowany do ochrony haseł.
    \item Wykorzystuje salt - losowe dane dodawane do hasła przed jego hashowaniem, co sprawia, że dwa identyczne hasła będą miały różne hashe.
    \item Zawiera parametr kosztu, który kontroluje, jak długo trwa proces hashowania. Wyższy koszt oznacza dłuższy czas hashowania, co zwiększa bezpieczeństwo przed atakami brute-force.
\end{itemize}

B-Crypt jest szczególnie ceniony za swoją elastyczność i możliwość dostosowania parametrów hashowania, co pozwala na zwiększenie bezpieczeństwa wraz ze wzrostem mocy obliczeniowej atakujących.

\textbf{Implementacja:}
\begin{lstlisting}[language=Python]
import bcrypt

# Hashowanie hasla
password = b"supersecret"
hashed = bcrypt.hashpw(password, bcrypt.gensalt())

# Weryfikacja hasla
if bcrypt.checkpw(password, hashed):
    print("Password matches")
else:
    print("Password does not match")
\end{lstlisting}

\subsection{SHA-256}
\textbf{Zastosowania:} SHA-256 jest szeroko używany w kryptowalutach (np. Bitcoin), cyfrowych podpisach, certyfikatach SSL oraz w innych zastosowaniach wymagających bezpiecznego hashowania danych.

\textbf{Działanie:}
\begin{itemize}
    \item SHA-256 należy do rodziny SHA-2 i jest zaprojektowany przez NSA.
    \item Produkuje 256-bitowe (32-bajtowe) hashe, co zapewnia wysoki poziom bezpieczeństwa.
    \item Algorytm jest odporny na ataki kolizyjne, co oznacza, że znalezienie dwóch różnych wejść dających ten sam hash jest bardzo trudne.
\end{itemize}

SHA-256 jest powszechnie używany ze względu na swoje bezpieczeństwo i wydajność. Jest kluczowym elementem w technologii blockchain, gdzie zapewnia integralność transakcji.

\textbf{Implementacja:}
\begin{lstlisting}[language=Python]
import hashlib

# Hashowanie wiadomosci
message = "hello world"
hashed = hashlib.sha256(message.encode()).hexdigest()

print(hashed)
\end{lstlisting}

\subsection{MD5}
\textbf{Zastosowania:} MD5 był kiedyś powszechnie używany do sprawdzania integralności plików i danych oraz przechowywania haseł, ale ze względu na swoje słabe strony, jest teraz uważany za przestarzały.

\textbf{Działanie:}
\begin{itemize}
    \item Produkuje 128-bitowe (16-bajtowe) hashe.
    \item Znany jest z podatności na kolizje, co oznacza, że jest możliwe znalezienie dwóch różnych wejść dających ten sam hash.
    \item Został zaprojektowany przez Ronalda Rivesta w 1991 roku jako następca algorytmu MD4.
\end{itemize}

MD5 jest obecnie uważany za niewystarczająco bezpieczny do większości zastosowań kryptograficznych. Pomimo to, jest nadal używany w niektórych starszych systemach ze względu na swoją prostotę i szybkość.

\textbf{Implementacja:}
\begin{lstlisting}[language=Python]
import hashlib

# Hashowanie wiadomosci
message = "hello world"
hashed = hashlib.md5(message.encode()).hexdigest()

print(hashed)
\end{lstlisting}

\subsection{SHA-1}
\textbf{Zastosowania:} SHA-1 był kiedyś szeroko używany do cyfrowych podpisów i certyfikatów SSL, ale z powodu znanych słabości bezpieczeństwa, jego użycie jest obecnie odradzane.

\textbf{Działanie:}
\begin{itemize}
    \item Produkuje 160-bitowe (20-bajtowe) hashe.
    \item Znany jest z podatności na kolizje, a ataki praktyczne na SHA-1 zostały zaprezentowane, co znacznie zmniejszyło zaufanie do jego bezpieczeństwa.
    \item SHA-1 został zaprojektowany przez NSA i opublikowany w 1993 roku.
\end{itemize}

Pomimo znanych problemów bezpieczeństwa, SHA-1 jest nadal używany w niektórych starszych systemach i aplikacjach, które nie zostały jeszcze zaktualizowane.

\textbf{Implementacja:}
\begin{lstlisting}[language=Python]
import hashlib

# Hashowanie wiadomosci
message = "hello world"
hashed = hashlib.sha1(message.encode()).hexdigest()

print(hashed)
\end{lstlisting}

\subsection{Blake2}
\textbf{Zastosowania:} Blake2 to nowoczesny algorytm hashowania używany do szerokiego zakresu zastosowań, takich jak podpisy cyfrowe, generowanie kluczy i inne zastosowania kryptograficzne.

\textbf{Działanie:}
\begin{itemize}
    \item Blake2 jest szybszy niż MD5, SHA-1 i SHA-256, a jednocześnie zapewnia wysoki poziom bezpieczeństwa.
    \item Dwa warianty: Blake2b (dla 64-bitowych systemów) i Blake2s (dla 32-bitowych systemów).
    \item Blake2 został zaprojektowany, aby być prostym w implementacji, a jednocześnie bardzo bezpiecznym i wydajnym.
\end{itemize}

Blake2 jest szeroko stosowany w nowoczesnych systemach kryptograficznych ze względu na swoją efektywność i bezpieczeństwo.

\textbf{Implementacja:}
\begin{lstlisting}[language=Python]
import hashlib

# Hashowanie wiadomosci
message = "hello world"
hashed = hashlib.blake2b(message.encode()).hexdigest()

print(hashed)
\end{lstlisting}

\section{Czas łamania algorytmów}

Poniższa tabela przedstawia szacowany czas łamania różnych algorytmów hashowania dla wyrażeń o różnej długości:

\begin{tabular}{|c|c|c|}
\hline
Algorytm & Czas łamania (8-znakowe haslo) & Czas łamania (16-znakowe haslo) \\
\hline
B-Crypt & Kilka lat & Miliony lat \\
SHA-256 & Kilka godzin & Kilka tysięcy lat \\
MD5 & Minuty & Kilka lat \\
SHA-1 & Godziny & Kilka tysięcy lat \\
Blake2 & Kilka godzin & Kilka tysięcy lat \\
\hline
\end{tabular}

Warto zauważyć, że czas łamania hasła zależy od wielu czynników, takich jak moc obliczeniowa atakującego, długość i złożoność hasła oraz dodatkowe zabezpieczenia, takie jak użycie soli.

\section{Wnioski}
Algorytmy hashowania różnią się pod względem bezpieczeństwa i wydajności. B-Crypt jest obecnie uważany za jeden z najbezpieczniejszych algorytmów do przechowywania haseł, jednak jego użycie wiąże się z wyższym kosztem obliczeniowym. SHA-256, choć bardzo bezpieczny, jest mniej odporny na ataki brute-force w porównaniu do B-Crypt. Algorytmy takie jak MD5 i SHA-1, kiedyś popularne, są teraz uważane za przestarzałe i niebezpieczne z powodu znanych podatności na kolizje. Blake2 oferuje nowoczesne podejście, łącząc wysoką wydajność z bezpieczeństwem.

\section{Referencje}
\begin{itemize}
    \item \url{https://www.overleaf.com/learn/latex/Learn_LaTeX_in_30_minutes}
    \item \url{https://www.dobreprogramy.pl/programy-dla-studentow-jak-rozpoczac-prace-z-latex-em,6628431494375041a}
    \item \url{https://maumneto.medium.com/git-vs-code-overleaf-91ecfd586b36}
    \item \url{https://en.wikipedia.org/wiki/Cryptographic_hash_function}
    \item \url{https://en.wikipedia.org/wiki/Bcrypt}
    \item \url{https://en.wikipedia.org/wiki/SHA-2}
    \item \url{https://en.wikipedia.org/wiki/MD5}
    \item \url{https://en.wikipedia.org/wiki/SHA-1}
    \item \url{https://en.wikipedia.org/wiki/BLAKE_(hash_function)}
\end{itemize}

\end{document}
